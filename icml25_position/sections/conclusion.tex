% 第7章:结论
\section{Conclusion}

In this position paper, we have argued that 
leveraging massive distributed edge devices can break barriers of data and computing wall, and everyone can participate in training large models with small edge devices.
Our comprehensive analysis demonstrated the vast untapped potential of edge resources, with smartphone data volume reaching approximately 33.1 EB and a combined computing power of around 9278 EFLOPS  in the past 5 years. 
These edge resources offer unique advantages in terms of data diversity, privacy, real-time context, and computing efficiency. 
% While significant challenges remain in managing heterogeneity and coordination across distributed systems, 
This paradigm shift towards distributed training could democratize AI development and open an exciting new chapter in the scaling of foundation models.

% In this position paper, we have argued that leveraging massive distributed 
% edge devices can break through current AI development bottlenecks by enabling 
% everyone to participate in training large models with small devices. By 
% analyzing challenges of data depletion and compute monopolization, we 
% demonstrated the vast untapped potential of distributed edge resources. 
% Through examining technical foundations in federated learning and distributed 
% architectures, we showed how these collective resources could democratize AI 
% development and push model scaling boundaries.

% Our analysis of edge computing revealed unique advantages in data diversity, 
% computational capacity, and energy efficiency, though significant challenges 
% remain in managing data heterogeneity and device coordination. The 
% environmental benefits of distributed training are particularly noteworthy, 
% as it eliminates the need for energy-intensive data centers while reducing 
% data transmission costs. Our analysis suggests that this paradigm shift 
% towards distributed training could democratize AI development and open an 
% exciting new chapter in the scaling of foundation models.